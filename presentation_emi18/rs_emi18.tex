%% Rimple Sandhu, Carleton University
%% rimple_sandhu@outlook.com

% Declare the theme
\documentclass[xcolor=dvipsnames,10pt]{beamer}
\renewcommand{\baselinestretch}{1.0} 
\usepackage{setspace}
% include the file with most used mathematical expressions 
\usepackage[makeroom]{cancel}
\usepackage{xfrac}
\usepackage{tikz}
\usetikzlibrary{shapes,arrows,chains,positioning,fit,calc}
\usepackage{cleveref}
\usepackage{verbatim}
\usepackage{rotating}
\usepackage{multirow}
\usepackage{amsfonts,amssymb,latexsym}
\usepackage{amsmath}
\usepackage{upgreek}
\colorlet{shadecolor}{gray!90}
%% mathematical symbols
\newcommand{\df}{\displaystyle\frac}
\newcommand{\ik}[1]{\textit{#1}}
\newcommand{\epar}[1]{\mbox{$e_{#1}$}}
\newcommand{\deter}[1]{\mbox{$|#1|$}}
\newcommand{\boldbb}[1]{\mbox{\boldmath $\mathbb {#1}$}}
\newcommand{\boldcal}[1]{\mbox{\boldmath $\mathcal {#1}$}}
\newcommand{\rtext}[1]{{\color{red}{#1}}}
\newcommand{\btext}[1]{{\color{blue}{#1}}}
\newcommand{\U}{\mathcal{U}}
\def\bvarepsilon{\mbox{\boldmath $\varepsilon$}}
\def\balpha{\mbox{\boldmath $\alpha$}}
\def\bbeta{\mbox{\boldmath $\beta$}}
\def\bsigma{\mbox{\boldmath $\sigma$}}
\def\bxi{\mbox{\boldmath $\xi$}}
\def\betaa{\mbox{\boldmath $\eta$}}
\def\brho{\mbox{\boldmath $\rho$}}
\def\bmu{\mbox{\boldmath $\mu$}}
\def\bnu{\mbox{\boldmath $\nu$}}
\def\bzeta{\mbox{\boldmath $\zeta$}}
\def\bpsi{\mbox{\boldmath $\psi$}}
\def\blambda{\mbox{\boldmath $\lambda$}}
\def\bnabla{\mbox{\boldmath $\nabla$}}
\def\btheta{\mbox{\boldmath $\theta$}}
 \def\bvarphi{\mbox{\boldmath $\varphi$}}
 \def\bkappa{\mbox{\boldmath $\kappa$}}
 
% Bold capital Greek letters
\def\bLambda{\mbox{\boldmath $\Lambda$}}
\def\bOmega{\mbox{\boldmath $\Omega$}}
\def\bOmhat{\mbox{\boldmath $\hat \Omega$}}
\def\bPhi{\mbox{\boldmath $\Phi$}}
\def\bPsi{\mbox{\boldmath $\Psi$}}
\def\bDelta{\mbox{\boldmath $\Delta$}}
\def\bTheta{\mbox{\boldmath $\Theta$}}
\def\bDelta{\mbox{\boldmath $\Delta$}}
\def\bSigma{\mbox{\boldmath $\Sigma$}}
\def\bVarphi{\mbox{\boldmath $\Varphi$}}

% Bold face capital letters
\def\Ex{\mbox{\text{E}}}
\def\bF{\mbox{\bf F}}
\def\bG{\mbox{\bf G}}
\def\bH{\mbox{\bf H}}
\def\bI{\mbox{\bf I}}
\def\bJ{\mbox{\bf J}}
\def\bK{\mbox{\bf K}}
\def\bQ{\mbox{\bf Q}}

\newcommand{\bbD}{\mathcal{\mbox{\bf D}}}
\newcommand{\cL}{\mathcal{L}}
\newcommand{\pL}{\mathcal{LP}}
\newcommand{\M}{\mathcal{M}}
\newcommand{\N}{\mathcal{N}}
\newcommand{\Q}{\mathcal{Q}}
\newcommand{\calS}{\mathcal{S}}
\def\bbM{\mbox{$\M$}}
%\def\N{\mbox{$\text{N}$}}
\def\LN{\mbox{$\text{LN}$}}
\def\Hess{\mbox{$\text{H}$}}
\def\R{\mathbb R}
\def\bbMnum{\mbox{$\bbM_{\text{num}}$}}
\def\th{\ensuremath{\theta}}
\def\dth{\ensuremath{\dot{\theta}}}
\def\ddth{\ensuremath{\ddot{\theta}}}
\def\dthh{\ensuremath{{\theta'}}}
\def\ddthh{\ensuremath{{\theta''}}}
\def\prob{\mbox{$\text{p}$}}
\def\Prob{\mbox{$\text{P}$}}
\def\moment{\mbox{$M_{\text{EA}}$}}
\def\bx{\mbox{\bf u}}
\def\bxj{\mbox{$\bx_j$}}
\def\bxja{\mbox{$\bx_j^a$}}
\def\bxjf{\mbox{$\bx_j^f$}}
\def\bxjp{\mbox{$\bx_{j+1}$}}
\def\bxjpf{\mbox{$\bx_{j+1}^f$}}
\def\bxjk{\mbox{$\bx_{j(k)}$}}
\def\bxjkf{\mbox{$\bx_{j(k)}^f$}}
\def\bxjkpf{\mbox{$\bx_{j(k+1)}^f$}}
\def\bxjka{\mbox{$\bx_{j(k)}^a$}}
\def\bxjkp{\mbox{$\bx_{j(k+1)}$}}
\def\bd{\mbox{\bf d}}
\def\by{\mbox{\bf y}}
\def\bdk{\mbox{$\bd_k$}}
\def\bdkp{\mbox{$\bd_{k+1}$}}
\def\bdtok{\mbox{$\bd_{1:k}$}}
\def\bdtokm{\mbox{$\bd_{1:k-1}$}}
\def\bdn{\mbox{$\bD$}}
\def\bq{\mbox{\bf q}}
\def\bqj{\mbox{$\bq_j$}}
\def\barqj{\mbox{$\bar{\bq}_j$}}
\def\bQj{\mbox{$\bQ_j$}}
\def\Mtrue{\mbox{$\M_{\text{true}}$}}

\def\np{\ensuremath{n_{\phi}}}
\def\nu{\ensuremath{n_{u}}}
\def\ns{\ensuremath{n_{\psi}}}
\def\nd{\ensuremath{n_{d}}}

\def\bA{\mbox{\bf A}}
\def\bB{\mbox{\bf B}}
\def\bC{\mbox{\bf C}}
\def\bD{\mbox{\bf D}}

\def\bAj{\mbox{$\bA_j$}}
\def\bBj{\mbox{$\bB_j$}}
\def\bCk{\mbox{$\bC_k$}}
\def\bDk{\mbox{$\bD_k$}}

\def\bg{\mbox{\bf g}}
\def\bh{\mbox{\bf h}}
\def\bgj{\mbox{$\bg_j$}}
\def\bhk{\mbox{$\bh_k$}}
\def\bff{\mbox{\bf f}}
\def\bffj{\mbox{$\bff_j$}}

\def\sphi{\ensuremath{\phi}}
\def\bphi{\mbox{\ensuremath{\upphi}}}
\def\beps{\mbox{\ensuremath{\upepsilon}}}
\def\bGamma{\mbox{\boldmath $\Gamma$}}
\def\bepsk{\mbox{$\beps_k$}}
\def\bGammak{\mbox{$\bGamma_k$}}
\def\barepsk{\mbox{$\bar{\beps}_k$}}
\def\bP{\mbox{\bf P}}
\def\bPja{\mbox{$\bP_j^a$}}
\def\bPjf{\mbox{$\bP_j^f$}}
\def\bPjpf{\mbox{$\bP_{j+1}^f$}}
\def\bPjk{\mbox{$\bP_{j(k)}$}}
\def\bPjkf{\mbox{$\bP_{j(k)}^f$}}
\def\bPjkpf{\mbox{$\bP_{j(k+1)}^f$}}
\def\bPjka{\mbox{$\bP_{j(k)}^a$}}
\def\tj{\mbox{$t_j$}}
\def\dtj{\mbox{$\Delta t_j$}}
\def\tjp{\mbox{$t_{j+1}$}}
\def\tjk{\mbox{$t_{j(k)}$}}
\def\dtjk{\mbox{$\Delta t_{j(k)}$}}
\def\tjkp{\mbox{$t_{j(k+1)}$}}
\newlength{\figwidthb}
\setlength{\figwidthb}{0.32\textwidth}
\newcommand{\beq}{\begin{equation}}
\newcommand{\eeq}{\end{equation}}
\newcommand{\seq}[1]{\begin{equation} \begin{split} #1 \end{split} \end{equation}}
\newcommand{\boxedseq}[1]{\begin{equation}\boxed{ \begin{split} #1 \end{split}} \end{equation}}

\def\Ex{\mbox{$\text{E}$}}
\def\bphis{\ensuremath{\bphi^*}}
\def\phis{\ensuremath{\phi^*}}
\def\smle{\ensuremath{\bpsi_{\text{mle}}}}
\def\smap{\ensuremath{\bpsi_{\text{map}}}}
\def\amap{\ensuremath{\alpha_{\text{map}}}}
\def\mle{\ensuremath{\bphi_{\text{mle}}}}
\def\map{\ensuremath{\bphi_{\text{map}}}}
\def\emap{\ensuremath{\hat{\bphi}_{\text{map}}}}
\def\bphid{\ensuremath{\bphi'}}
\def\bphidd{\ensuremath{\bphi''}}
\def\mud{\mbox{$\mu_f$}}
\def\sigd{\mbox{$\sigma_f$}}
\def\bphig{\ensuremath{\bphi^{(g)}}}
\def\bphij{\ensuremath{\bphi^{(j)}}}
\def\bphik{\ensuremath{\bphi_{\text{-}\psi}}}
\def\bphiu{\ensuremath{\bphi_{\psi}}}
\def\phig{\ensuremath{\phi^{(g)}}}
\def\phij{\ensuremath{\phi^{(j)}}}
\def\phii{\ensuremath{\phi^{(i)}}}
\def\Rey{\ensuremath{\text{Re}_{c}}}
\def\Cm{\ensuremath{\text{C}_{m}(\th)}}
\def\ths{\ensuremath{\bphi^{\star}}}
\def\ts{\ensuremath{\phi^{\star}}}
\def\d{\mbox{d}}
\def\dt{\ensuremath{\Delta t}}
\def\ACT{\ensuremath{\tau_\text{int}}}
\def\eACT{\ensuremath{\hat{\tau}_\text{int}}}
\def\eACTm{\ensuremath{\hat{\tau}_\text{int}^{\text{max}}}}
\def\bphiad{\ensuremath{\bphi_{\text{ad}}}}
\def\IEA{\ensuremath{I_{\text{EA}}}}

\let\originaleqref\eqref
\renewcommand{\eqref}{Eq.~\originaleqref}

% added for unsteady
\def\Mac{\mbox{$M_{ac}$}}
\def\Mea{\mbox{$M_{ea}$}}
\def\dMea{\mbox{$\dot{M}_{ea}$}}
\def\ddMea{\mbox{$\ddot{M}_{ea}$}}
\def\dddMea{\mbox{$\dddot{M}_{ea}$}}
\def\dddth{\mbox{$\dddot{\theta}$}}
\def\ddddth{\mbox{$\ddddot{\theta}$}}
\def\dMm{\mbox{$\dot{M}$}}
\def\w{\mbox{$\omega$}}
\def\wf{\mbox{$\omega_f$}}
\def\dCm{\mbox{$\dot{C}_{M}$}}
\def\ddCm{\mbox{$\ddot{C}_{M}$}}
\def\dddCm{\mbox{$\dddot{C}_{M}$}}
\def\cLift{\ensuremath{L}}
\def\half{\mbox{$\df{1}{2}$}}
\def\Cm{\ensuremath{C_{M}}}
\def\ah{\mbox{$a_h$}}
\def\metre{\mbox{$\text{ m}$}}
\def\Uinf{\mbox{$U$}}
\def\wth{\mbox{$\omega_{\theta}$}}
\def\rth{\mbox{$r_{\theta}$}}
\def\wcl{\mbox{$\omega_{c}$}}
\def\dn{\mbox{$d_{1:n}$}}
\def\rcl{\mbox{$r_{c}$}}
\def\wdv{\mbox{$\text{w}_{3c/4}$}}
\def\qtil{\mbox{$\tilde{q}$}}

\newcommand{\dder}[2]{\ensuremath{\df{d^2 #1}{d{#2}^2}}}
\newcommand{\der}[2]{\ensuremath{\df{d #1}{d{#2}}}}
\newcommand{\expn}[1]{\ensuremath{\exp \left\{{#1}\right\} } }


\usepackage{cleveref}
\usepackage{tikz}
\usetikzlibrary{shapes,arrows,chains,positioning,fit,calc}
% The theme
% options: CambridgeUS, default, Boadilla, Madrid, Singapore, Copenhagen
\usetheme{CambridgeUS}

% your block environment formatting
% options: lily (Default), orchid, rose
\usecolortheme{rose}

% Your frametitle and footline colouring
% options: whale, seahorse, dolphin, beaver, crane, dove, seagull
\usecolortheme{seahorse}
   
% Change the geometry of items in itemize list
% Options: ball, circle, rectangle, default
\setbeamertemplate{items}[circle]

% no shadow
\setbeamertemplate{blocks}[rounded][shadow=false]

% colour of block headline and bullet
% see http://www.math.umbc.edu/~rouben/beamer/quickstart-Z-H-25.html#node_tag_Temp_50
% some good colors: RawSienna, Sepia, BrickRed
\usecolortheme[named=Black]{structure}

% getting rid of navigation symbols on the slide on botton right corner
% uncomment if you want those
\setbeamertemplate{navigation symbols}{} 

% text font
%\usefonttheme{serif}

% Explicitly format the headline, footline and frametitle
\setbeamertemplate{headline}{}
\setbeamertemplate{frametitle}{
\begin{beamercolorbox}[wd=\paperwidth, ht=0.4in]{frametitle}
\hbox{\hspace{0.2in}\insertframetitle}\\[-0.1in] \rule{\paperwidth}{2pt}
\end{beamercolorbox}}

%% user defined environment for ease in coding
% for new figure
\newcommand{\newfigure}[2]{\begin{figure}\centering 
\includegraphics[width=#1\linewidth]{figs/#2}\end{figure}\vspace{-0.2in}}
% for centered alert
\newcommand{\alertb}[1]{\begin{center}\alert{#1}\end{center}}
\newcommand{\red}[1]{{\textcolor{red}{#1}}}
\newcommand{\blue}[1]{{\textcolor{blue}{#1}}}
% for itemize block
\newenvironment{newblockc}[1]
{\begin{block}{#1}\begin{itemize}\small
  \setlength{\itemsep}{4pt}}
{\end{itemize}\end{block}}
% items of newblockc
\newcommand{\itb}[1]{\item #1}
% Block with no bulleted list
\newcommand{\newblockb}[2]{\begin{block}{#1} \small #2 \end{block}}
% equation shortcuts
\newcommand{\eqa}[1]{\vspace{0.05in} \begin{center} $#1$ \end{center} \vspace{0.05in}}

%% presentation details needed in footline and title page generation
\title[EMI 2018]{\normalsize Automatic relevance determination priors in Bayesian model selection: Application to nonlinear fluid-structure interaction systems}
%\subtitle{Include Only If Paper Has a Subtitle}
\author[R.~Sandhu \textit{et al.}] % (optional, use only with lots of authors)
{\normalsize R.~Sandhu\inst{1}, C.~Pettit\inst{2}, M.~Khalil\inst{1,3},  A.~Sarkar\inst{1} and D.~Poirel\inst{4}}
% - Give the names in the same order as the appear in the paper.
% - Use the \inst{?} command only if the authors have different affiliation.

\institute[] % (optional, but mostly needed)
{
  \inst{1}%
  Carleton University, Ottawa, ON, Canada
      \and
  \inst{2}%
  United States Naval Academy, Annapolis, MD, USA
    \and
    \inst{3}%
Sandia National Laboratories, Livermore, CA, USA
  \and
  \inst{4}%
  Royal Military College of Canada, Kingston, ON, Canada

\vspace{0.02in}

  }
% - Use the \inst command only if there are several affiliations.
% - Keep it simple, no one is interested in street address.

\date[May 29 - June 1, 2018] % (optional, should be abbreviation of conference name)
{{\begin{center}\footnotesize EMI 2018 \\
Boston, MA, USA \\
May 29 - June 1, 2018\end{center}}
{\tiny \it 
Sandia National Laboratories is a multimission laboratory managed and operated by National Technology and Engineering Solutions of Sandia, LLC., a wholly owned subsidiary of Honeywell International, Inc., for the U.S. Department of Energy's National Nuclear Security Administration under contract DE-NA-0003525.
}}


% This is only inserted into the PDF information catalog. Can be left out. 
\subject{Non-deterministic approaches in aeroelasticity}

% Delete this, if you do not want the table of contents to pop up at
% the beginning of each subsection:
\AtBeginSubsection[]
{
  \begin{frame}<beamer>{Outline}
    \tableofcontents[currentsection,currentsubsection]
  \end{frame}
}


% If you wish to uncover everything in a step-wise fashion, uncomment
% the following command: 
%\beamerdefaultoverlayspecification{<+->}


\begin{document}

\begin{frame}[plain]
\titlepage
\end{frame}


\begin{frame}{Outline}
  \tableofcontents
  % You might wish to add the option [pausesections]
\end{frame}


\section{Problem definition}

%\begin{frame}{Introduction}
%\begin{newblockc}{Why model selection?}
%\itb{When dealing with nontrivial physics under limited a priori understanding of the system, multiple plausible models can be envisioned to represent the system with a reasonable accuracy.}
%\itb{A complex model overfits the data but results in a higher model prediction uncertainty.}
%\itb{A simpler model misfits the data but results in a lower model prediction uncertainty.}
%\itb{An “optimal” model provides a balance between data-fit and prediction uncertainty.}
%\end{newblockc}
%\end{frame}

\begin{frame}{Problem definition}
\begin{minipage}[b]{0.48\linewidth}\begin{newblockc}{Nonlinear aeroelastic oscillator}
\itb{Pure pitch (dof: 1) limit cycle oscillations (LCO) of a 2-D rigid airfoil in a transitional Re regime}
%\itb{Pure pitch or $\th(t)$ motion only}
\end{newblockc}
\end{minipage}
\begin{minipage}[b]{0.48\linewidth}
\newfigure{0.9}{wing.eps}
\end{minipage}
\newfigure{0.85}{raw_8p5.eps} \vspace{0.2in}
Goal: Identify the nature of unsteady and nonlinear aerodynamics causing the LCO by assimilating the noisy wind-tunnel observations.
\end{frame}

\begin{frame}{Problem definition}
\newblockb{Candidate model set: $\{\M_1, \M_2, \M_3, \M_4 \}$}{
Eq. of motion (Known): \\
$
\IEA \ddth + D \dth + K \th + K' \th^3  = D' \text{sign}(\dth) + \df{1}{2}\rho U^2c^2s\, \red{\Cm(\th,\dth,\ddth)}$ \\[0.1in]
Possible models of aerodynamics ($\Cm$): \\[0.1in]
$\M_1$ : $Cm = \epar{1} \th + \epar{2} \dth + \epar{3} \th^3 + \epar{4} \th^2 \dth + \blue{\sigma \xi(\tau)}$ \\[0.1in]
$\M_2$ : $\Cm = \epar{1} \th + \epar{2} \dth + \epar{3} \th^3 + \epar{4} \th^2 \dth +  \epar{5} \th^5 +  \epar{6} \th^4 \dth  + \blue{\sigma \xi(\tau)}$ \\[0.1in]
$\M_3$ : $ \df{\dCm}{B} + \Cm = \epar{1} \th + \epar{2} \dth + \epar{3} \th^3 + \epar{4} \th^2 \dth   + \df{c_6}{B} \ddth  + \blue{\sigma \xi(\tau) }$\\[0.1in]
$\M_4$ : $ \df{\dCm}{B} + \Cm = \epar{1} \th + \epar{2} \dth + \epar{3} \th^3 + \epar{4} \th^2 \dth +  \epar{5} \th^5 +  \epar{6} \th^4 \dth  + \df{c_6}{B} \ddth  + \blue{\sigma \xi(\tau)} $ \\[0.1in]

Measurement equation:
\beq
d_k = \th_k + \epsilon_k \;\; ; \;\;\; k = 1, \ldots, \nd
\eeq
}
\alertb{Bayesian model selection in discrete model space}
\end{frame}


\begin{frame}{Problem definition}
\newblockb{Bayesian model selection in discrete model space}{ Given observational data $\bdn$ = $\{\bd_1,\bd_2$, $\ldots, \bd_{\nd}\}$ and a candidate model set $\bbM$ = $ \{\M_1,\M_2\ldots\M_i\ldots\M_P\}$, the posterior model probability is calculated as 
\beq
\Prob(\M_i |\bdn,\bbM) = \df{\prob(\bdn |\M_i) \Prob(\M_i |\bbM)}{\prob(\bdn |\bbM)}
\eeq 
where $\prob(\bdn |\M_i)$ is the model evidence, which embodies the principle of Ockham's razor,
\beq
\underbrace{\ln \prob(\bdn|\M_i)}_\text{Log-evidence} =  \int \prob(\bdn|\bphi) \prob(\bphi) d \bphi = \;\;\underbrace{\Ex[\ln \prob(\bdn|\bphi,\M_i)]}_\text{Goodness-of-fit} - \underbrace{\Ex\left[\ln\df{\prob(\bphi|\bdn,\M_i)}{\prob(\bphi|\M_i)}\right]}_\text{Information gain (EIG)}
\eeq}
\textcolor{shadecolor}{ 
Sandhu \textit{et al.}, CMAME, 2017. \\
Sandhu \textit{et al.}, JCP, 2016. \\
Sandhu \textit{et al.}, CMAME, 2014.\\
Khalil \textit{et al.}, JSV, 2013}
\end{frame}


\begin{frame}{Problem definition}
\begin{newblockc}{Practical hurdles in implementing Bayesian model selection in discrete model space}
\itb{Sensitivity of parameter prior distribution to the posterior model probability or the model evidence}
\itb{Missing out on better candidate models}
\end{newblockc}
\alertb{Solution: Automatic relevance determination (ARD)}
\end{frame}

\section{Automatic Relevance Determination (ARD)}


\begin{frame}{Automatic Relevance Determination (ARD)}
\begin{newblockc}{Reformulating  the model selection problem:}
\itb{An encompassing model $\M$:\\[0.1in]  $\df{\dCm}{B} + \Cm = a_1 \th + a_2 \dth + a_3 \th^3 + a_4 \th^2 \dth +  a_5 \th^5 +  a_6 \th^4 \dth  + \df{c_6}{B} \ddth  + \sigma \xi(\tau)$}
\itb{The question we ask: Given measurements $\bdn$, find the optimal model nested under the overly-complicated encompassing model?}
\end{newblockc}
\end{frame}


\begin{frame}{Automatic Relevance Determination (ARD)}
\newblockb{Physics-driven + Data-driven + Prior knowledge}{Hybrid approach for assigning prior distributions by categorizing parameters based on prior knowledge about the aerodynamics as \blue{Required} (\bphik) or \red{Contentious} (\bphiu)\\ 
$\df{\dCm}{\blue{B}} + \Cm = \blue{a_1} \th + \blue{a_2} \dth + \red{a_3} \th^3 + \red{a_4} \th^2 \dth +  \red{a_5} \th^5 +  \red{a_6} \th^4 \dth  + \df{c_6}{B} \ddth  + \blue{\sigma} \xi(\tau)$}
\vspace{0.1in}
{
\renewcommand{\arraystretch}{2.0}
%\renewcommand{\tabcolsep}{1pt}
\begin{table}[!htbp]
\small
\centering
\begin{tabular}{|c||c|}
\hline
\multicolumn{2}{|c|}{Prior pdf, $\prob(\bphi|\bpsi)$ = $\prob(\bphik)\prob(\bphiu|\bpsi)$} \\
\multicolumn{2}{|c|}{Hyper-parameter, $\bpsi = \{\alpha_1, \alpha_2, \alpha_3, \alpha_4\}$ } \\ \hline
\blue{$\prob(\bphik)$} &  $ \cL(B|0.2, 50) \; \U(a_1| {-2},0) \; \U(a_2| {-2},0) \; \cL(\sigma |0.002,50) $  \\
\red{$\prob(\bphiu|\bpsi)$} & ARD prior, $\N\left(a_3|0,\frac{1}{\alpha_1}\right)\N\left(a_4|0,\frac{1}{\alpha_2}\right)\N\left(a_5|0,\frac{1}{\alpha_3}\right)\N\left(a_6|0,\frac{1}{\alpha_4}\right)$   \\ \hline 
\end{tabular}
\label{tab:t3}
\end{table}
}
\end{frame}


\begin{frame}{Automatic Relevance Determination (ARD)}
%\begin{newblockc}{Relevance through ARD prior}
%\itb{$\alpha_i = \infty$ means both prior and posterior pdf are a dirac delta function at zero. }
%\itb{All possible models nested under $\M$ can be construed by varying $\bpsi$ }
%\end{newblockc}
\begin{newblockc}{Using hierarchical Bayes approach:}
\itb{Posterior pdf $\prob(\bpsi|\bd)$ of hyper-parameter vector $\bpsi$,
\beq
\prob(\bpsi|\bd) = \df{\prob(\bd|\bpsi) \prob(\bpsi)}{\prob(\bd)}
\eeq}
\itb{Assuming flat prior for $\prob(\bpsi)$ \red{Task: Stochastic optimization},
\beq
\smap = arg  \max_{\bpsi} \{\prob(\bdn|\bpsi)\}
\eeq}
\itb{Model evidence as a function of hyper-parameter \red{Task: Evidence computation},
\beq \label{eq:evidence_int} \prob(\bdn|\bpsi) = \int \prob(\bdn|\bphi) \prob(\bphi|\bpsi) d \bphi 
\eeq}
\itb{Likelihood computation \red{Task: State estimation},
\beq 
\prob(\bdn|\bphi) = \prod_{k=1}^{\nd} \int \prob(\bdk|\bxjk,\bphi) \prob(\bxjk|\bdtokm,\bphi) d
\bxjk
\eeq}
\end{newblockc}
%\alertb{Optimal nested model $\equiv$ Encompassing model with prior $\prob(\bphi|\smap)$}
\end{frame}

\begin{frame}{Automatic Relevance Determination (ARD)}
\begin{newblockc}{Numerical implementation}
\itb{Evidence optimization: Derivative-free methods including line-search, \red{pattern search}, simplex method, evolutionary algorithms; and many others.}
\itb{Evidence computation: \red{Chib-Jeliazkov method}, Transitional MCMC, Power posteriors, Nested sampling, Annealed importance sampling, Harmonic mean estimator, Gauss-Hermite quadrature; and many others.}
\itb{MCMC sampler for Chib-Jeliazkov method: Metropolis-Hastings, Gibbs, TMCMC, adaptive Metropolis, \red{Delayed Rejection Adaptive Metropolis(DRAM)}; and many others}
\itb{State estimation: Kalman filter, \red{Extended Kalman filter}, unscented Kalman filter, ensemble Kalman filter, particle filter; and many others. }
\end{newblockc}
\end{frame}


\section{Numerical results: Unidimensional ARD}
\begin{frame}{Numerical results: Unidimensional ARD}
\newblockb{ARD prior for a relevant parameter}{Model proposed same as the data-generating model:
\beq \label{eq:aODE}
\df{\dCm}{\blue{B}} + \Cm = \blue{a_1} \th + \blue{a_2} \dth + \red{a_3} \th^3 + \blue{a_4} \th^2 \dth + \df{c_6}{\blue{B}} \ddth  + \blue{\sigma} \xi(\tau) ,
\eeq
Case 1: Gaussian ARD prior:
\beq
\prob(\bphi|\bpsi) = \cL(B|0.2, 50) \U(a_1| {-2},0)\U(a_2| {-2},0) \; \red{\N(a_3|0,1/\alpha)} \; \U(a_4|{-600},0)\cL(\sigma |0.002,50)
\eeq
Case 2: Laplace ARD prior:
\beq
\prob(\bphi|\bpsi) = \cL(B|0.2, 50) \U(a_1| {-2},0)\U(a_2| {-2},0) \; \red{\pL(a_3|0,1/\alpha)} \; \U(a_4|{-600},0)\cL(\sigma |0.002,50)
\eeq}
\newfigure{0.25}{gauss_vs_laplace.eps}
\end{frame}

\begin{frame}{Numerical results: Unidimensional ARD}
\newblockb{Observations:}{
- Change in log-evidence driven by loss of goodness-of-fit due to removal of $a_3$\\
- Log-evidence has higher slope near maxima and is minimally sloped elsewhere\\
- Both Laplace prior and Gaussian prior results in same parameter sparsity level.   
}
\newfigure{0.65}{uni/fig_E01.eps}
\end{frame}

%
%\begin{frame}[fragile]{Numerical results: Unidimensional ARD}
%\newblockb{ARD prior for a relevant parameter}{Evidence optimization using Pattern search algorithm from DAKOTA library\\
%- method =  asynch\_pattern\_search \\
%- initial\_delta = 3.0 \\
%- contraction\_factor = 0.5 \\
%- threshold\_delta = 1.0 (\red{Large enough to overcome sampling error!})}
%\small
%\begin{verbatim}
%eval_id_interface             x1         obj_fn 
%1            NO_ID             -8   -27810.12249 
%2            NO_ID            -11   -27806.66866 
%3            NO_ID             -5   -27813.56397 
%4            NO_ID             -2   -27776.23808 
%5            NO_ID           -3.5   -27813.59085 
%6            NO_ID           -1.5   -27733.59679 
%7            NO_ID           -5.5   -27812.92908 
%8            NO_ID           -2.5   -27802.01208 
%9            NO_ID           -4.5    -27813.8383 
%10           NO_ID           -6.5   -27811.79282
%<<<<< Best parameters          = -4.5000000000e+00
%<<<<< Best objective function  = -2.7813838302e+04
%\end{verbatim}
%\end{frame}


\begin{frame}{Numerical results: Unidimensional ARD}
\begin{center}
Effect of using zero-mean ARD priors on parameter estimates
\end{center}
\begin{figure}[!htbp]
\centering
\includegraphics[width=\figwidthb]{figs/uni/mpdf_4_FLAT.eps}
\includegraphics[width=\figwidthb]{figs/uni/mpdf_4_GAUSS.eps} 
\includegraphics[width=\figwidthb]{figs/uni/mpdf_4_LAP.eps}\\
\caption{Comparison of marginal posterior pdf of parameter $a_3$ obtained using optimized Gaussian and Laplace ARD prior, compared with the marginal posterior obtained using a flat prior for parameter $a_3$. }
\label{fig:md3}
\end{figure}
\end{frame}

\begin{frame}{Numerical results: Unidimensional ARD}
\newblockb{ARD prior for an irrelevant parameter}{Model proposed has an additional term than the data-generating model: \\
\beq \label{eq:aODE}
\df{\dCm}{B} + \Cm = a_1 \th + a_2 \dth + a_3 \th^3 + a_4 \th^2 \dth + \red{a_5\th^5} + \df{c_6}{B} \ddth  + \sigma \xi(\tau) ,
\eeq 
\seq{
\prob(\bphi|\bpsi) = & \cL(B|0.2, 50) \U(a_1| {-2},0)\U(a_2| {-2},0) \U(a_3|{-250},250)   \\
&\; \U(a_4|{-600},0) \red{\N(a_5|0,1/\alpha)} \cL(\sigma |0.002,50)
}
Observations: \\
- The change in log-evidence is driven by the decrease in Complexity (EIG) due to the removal of irrelevant parameter. \\
- Log-evidence is flat in regions higher the optimal hyperparameter}
\newfigure{0.8}{uni/fig_02.eps}
\end{frame}


%
%\begin{frame}[fragile]{Numerical results: Unidimensional ARD}
%\newblockb{ARD prior for an irrelevant parameter}{Evidence optimization using Pattern search algorithm from DAKOTA library\\
%- method =  asynch\_pattern\_search \\
%- initial\_delta = 3.0 \\
%- contraction\_factor = 0.5 \\
%- threshold\_delta = 1.0 (\red{Large enough to overcome sampling error})\\
%- Maximum bound = 4.0 \red{(Important!)}}
%\small
%\begin{verbatim}
%%eval_id interface             x1         obj_fn 
%1            NO_ID             -8   -27811.67396 
%2            NO_ID            -11   -27808.20306 
%3            NO_ID             -5   -27813.59631 
%4            NO_ID             -2   -27813.28177 
%5            NO_ID           -3.5   -27813.45931 
%6            NO_ID           -6.5   -27813.24228 
%<<<<< Best parameters          = -5.0000000000e+00 x1
%<<<<< Best objective function  = -2.7813596314e+04
%\end{verbatim}
%\end{frame}



\section{Numerical results: Multidimensional ARD}
\begin{frame}{Numerical results: Multidimensional ARD}

\newblockb{Model proposed:}{
\beq \label{eq:mdODE}
\blue{\df{\dCm}{B}} + \Cm = \blue{a_1 \th} + \blue{a_2 \dth} + \red{a_3 \th^3} + \red{a_4 \th^2 \dth} +  \red{a_5 \th^5} +  \red{a_6 \th^4 \dth}  + \blue{\df{c_6}{B} \ddth}  + \blue{\sigma \xi(\tau)}
\eeq}
{
\renewcommand{\arraystretch}{2.0}
%\renewcommand{\tabcolsep}{1pt}
\begin{table}[!htbp]
\small
\centering
\begin{tabular}{|c||c|}
\hline
\multicolumn{2}{|c|}{Prior pdf, $\prob(\bphi|\bpsi)$ = $\prob(\bphik)\prob(\bphiu|\bpsi)$} \\
\multicolumn{2}{|c|}{Hyper-parameter, $\bpsi = \{\alpha_1, \alpha_2, \alpha_3, \alpha_4\}$ } \\ \hline
\blue{$\prob(\bphik)$} &  $ \cL(B|0.2, 50) \; \U(a_1| {-2},0) \; \U(a_2| {-2},0) \; \cL(\sigma |0.002,50) $  \\
\red{$\prob(\bphiu|\bpsi)$} & ARD prior, $\N\left(a_3|0,\frac{1}{\alpha_1}\right)\N\left(a_4|0,\frac{1}{\alpha_2}\right)\N\left(a_5|0,\frac{1}{\alpha_3}\right)\N\left(a_6|0,\frac{1}{\alpha_4}\right)$   \\ \hline 
\end{tabular}
\label{tab:t3}
\end{table}
}
\end{frame}



\begin{frame}{Numerical results: Multidimensional ARD}
\newfigure{0.85}{fig_04.eps}
%\alertb{Output of evidence maximization}
\newfigure{0.65}{fig_01.eps}
\end{frame}



\begin{frame}{Numerical results: Multidimensional ARD}
\begin{figure}[!htbp]
\centering
\includegraphics[width=0.97\figwidthb]{figs/mpdf_4.eps}
\includegraphics[width=0.97\figwidthb]{figs/mpdf_5.eps}
\includegraphics[width=1.1\figwidthb]{figs/jpdf_4_5.eps}\\
\includegraphics[width=0.97\figwidthb]{figs/10_PE_U1_OPTIMAL/mpdf_4.eps}
\includegraphics[width=0.97\figwidthb]{figs/10_PE_U1_OPTIMAL/mpdf_5.eps}
\includegraphics[width=1.1\figwidthb]{figs/10_PE_U1_OPTIMAL/jpdf_4_5.eps}
\caption{Comparison of marginal and joint posterior pdf of relevant parameters $a_3$ and $a_4$ for ARD prior with optimal hyper-parameters and flat priors pdf. }
\label{fig:md3}
\end{figure}
\end{frame}

\begin{frame}{Conclusion}
\begin{newblockc}{Conclusion}
\itb{The concept of automatic relevance determination (ARD) is exploited as an automatic model selection tool with application to nonlinear dynamical systems modelled using stochastic ordinary differential equations (ODE).}
\itb{ARD approach is validated using a synthetically generated nonlinear aeroelastic oscillations.}
\itb{Both Laplace and Gaussian ARD prior produced same parameter sparsity level.}
\itb{Derivative-free optimization techniques with bound constraint are well-suited for optimizing model evidence due to the flatness of objective function (Log-evidence) away from maxima.}
\end{newblockc}

\begin{newblockc}{Future direction}
\itb{Using gradient/hessian information to expedite the optimization of model evidence.}
\itb{Comparing the ARD approach to LASSO/Ridge regression techniques.}
\end{newblockc}

{\begin{center} \footnotesize R. Sandhu, C. Pettit, M. Khalil, D. Poirel, A. Sarkar, Bayesian model selection using automatic relevance determination for nonlinear dynamical systems, Computer Methods in Applied Mechanics and Engineering (2017).\end{center}}

\end{frame}

\begin{frame}{Acknowledgement}
 \alertb{Thank you.}
 
\begin{newblockc}{Financial contributions}
\itb{Department of National Defence, Canada}
\itb{Ontario Graduate Scholarship program}
 \itb{Carleton University, Ottawa, Canada}
 \itb{Natural Sciences and Engineering Research Council, Canada}
\end{newblockc}
\begin{newblockc}{Supercomputers used}
 \itb{HP Linux Cluster, Carleton University, Ottawa, Canada}
 \itb{CLUMEQ, McGill University, Montreal, Canada}
 \itb{SciNet, University Of Toronto, Toronto, Canada}
\end{newblockc}

\begin{newblockc}{External libraries used}
 \itb{Dakota, UQ Toolkit (UQTk) [Developed by Sandia National Lab]}
 \itb{Armadillo (Linear algebra library for C++)}
\end{newblockc}

\end{frame}


%\begin{frame}
%\begin{center}
%{\large \alert{Thank you.} \\[0.3in]  \alert{Questions?}}\\[1.0in]
%{``Life is the art of reaching sufficient conclusions \\ from ins ff ci nt d ta"}
%\end{center}
%\end{frame}
%
%
%%% other helpful stuff
%
%\begin{frame}{Comparison of evidence calculation methods}
%\begin{newblockc}{Proposed model}
%\itb{Same as the data-generating model:
%\beq \label{eq:num_model1}
%\ddth = a_1 \th + a_2 \dth + a_3 \th^2 + a_4 \th^3 + a_5 \th^2 \dth + \sigma \xi(t)
%\eeq}
%\itb{Joint prior pdf:
%\seq{ 
%\prob(\bphi|\bpsi) = \prob(\bphi) &= \prob(a_1) \prob(a_2) \prob(a_3)\prob(a_4)  \prob(a_5) \prob(\sigma) \\
%& = \text{U(-10$^{3}$, 0) U(0, 8) U(-10$^{3}$,0) U(0,10$^{5}$) U(-10$^{-4}$,0) LN(0.002,0.5)}
%}}
%\end{newblockc}
%{
%\renewcommand{\arraystretch}{1.1}
%\begin{table}[!htbp]
%\centering
%\begin{tabular}{@{\extracolsep{\fill}}|c|c|c|c|}
%\hline
% Method & Chib-Jeliazkov & Laplace method & Harmonic mean\\ \hline
% log-evidence (10$^{4}$) & 2.33547 & 2.33548 & 2.34247 \\ \hline
%\end{tabular}
%\caption{ Comparison of evidence estimates from different methods}
%\label{tab:logevidence}
%\end{table}
%}
%\end{frame}
%
%\begin{frame}{Evidence maximization algorithm}
%\begin{algorithmic}[1]
%\State Initiate the MAP estimates $\smap$ and $\map$ and the RWM proposal covariance $C$.
%\While{optimization stopping criterion is not met} 
%  \State Propose a new $\bpsi$ using line-search algorithm
%  \State Initiate the MCMC sampler using $\map$ and $\bSigma$ from previous iteration.
%  \While{chain stopping criterion is not met}
%    \State Generate W number of MCMC samples on each core.
%    \State Update $\map$ at each accepted Markov chain move.
%    \State Calculate the sample covariance $\bSigma$ using samples from all cores.
%    \State Update the proposal covariance $C$ using $\bSigma$
%    \State Update $\map$ as the best of all cores.
%  \EndWhile
%  \State Calculate evidence by equating the new $\map$ and $\bSigma$ in \eqref{eq:laplace}. 
%  \If {New evidence optimum is found} Update $\smap$
%  \EndIf
%\EndWhile
%  \end{algorithmic}
%\end{frame}
%
%\begin{frame}{Laplace's method of calculating evidence}
%\beq \label{evidence_int}
%\prob(\bdn|\bphi) =  \prod_{k=1}^{\nd} \int \prob(\bdk|\bxjk,\bphi) \prob(\bxjk|\bdtokm,\bphi) d\bxjk
%\eeq
%\beq \label{eq:logevid}
%\ln \prob(\bdn|\bpsi) = \ln \left[ \int \exp \{\cL(\bphi,\bpsi)\} d\bphi\right] 
%\eeq
%\beq \label{eq:taylor}
%\cL(\bphi,\bpsi) \approx \cL(\map,\bpsi) - \half (\bphi - \map)^T \bSigma(\bpsi)^{-1}  (\bphi - \map)
%\eeq
%\beq
%\Hess(\bpsi) = - \bSigma(\bpsi)^{-1} = \df{\partial^2 \cL(\bphi,\bpsi)}{\partial \bphi\bphi^T}\, \bigg|_{\displaystyle \bphi =\map}  .
%\eeq
%\end{frame}
%
%\begin{frame}{Laplace's method of calculating evidence (Contd.)}
%\seq{
%\ln \prob(\bdn|\bpsi) & \approx \ln \left[ \int \exp \left\{ \cL(\map,\bpsi) - \half (\bphi - \map)^T \bSigma(\bpsi)^{-1}  (\bphi - \map)\right\} d\bphi\right] \\
%& \approx \ln \left[ \int \exp \{\cL(\map,\bpsi)\} \exp  \left\{- \half (\bphi - \map)^T \bSigma(\bpsi)^{-1}  (\bphi - \map)\right\} d\bphi\right] \\
%& \approx \ln\left[ \{ \exp \{\cL(\map,\bpsi)\}\right] + \ln \left[ \int \exp  \left\{- \half (\bphi - \map)^T \bSigma(\bpsi)^{-1}  (\bphi - \map)\right\} d\bphi\right] \\
%& \approx \cL(\map,\bpsi) + \ln   \sqrt{\df{(2 \pi)^{\np}}{\deter{\bSigma(\bpsi)^{-1}} }} \\
%& \approx \cL(\map,\bpsi) + \df{\np}{2} \ln 2 \pi - \half \ln\deter{\bSigma(\bpsi)^{-1}}  \\ 
%& \approx \cL(\map,\bpsi) + \df{\np}{2} \ln 2 \pi + \half\ln\deter{\bSigma(\bpsi)} \label{eq:laplace}
%}
%\end{frame}


\end{document}
